The system focuses on simplifying the management of information related to train traveling and ticket payment for its users. The system will provide an account for each user which will store information regarding travel distance, destinations, payments, etc. This will help both users travelling by train and the companies providing the travelling services by keeping track of all these components in a centralized manner. 

The system will present the user with the choice of making an automated payment for each trip through the connection between the mobile app and the beacon in the train cart. Alternatively, an external payment system will be presented to the user I every train station where they can log in and perform the transaction. 

A second important target for our system will be represented by companies providing travelling services by train, which may include governmental institutions or different private companies.

\section{Business drivers}
This chapter describes the business forces acting on the system: most important revenue streams and costs and give a justification for the system.

\subsection{Revenue streams}
The main revenue stream for the system will consist of the travel service providers. In order to use the Easy Tickting system, they will have to pay a monthly fee which will include storage and maintenance costs. 

Depending on how well the system usage evolves over time, once enough travel service providers decide to use the system, a small monthly fee can be charged to the user as well. Also, the mobile app functionalities can be improved over time and be accessed by the user over a pay-by-use scenario(e.g. check distances travelled over the last period of time, check fees, see a list of possible price reductions for similar journeys).

\subsection{Costs} 


