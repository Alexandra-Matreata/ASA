%!TEX root = main.tex

In this section the main Stakeholders and the functional and non functional requirements of Easy Ticketing will be introduced.

\section{Stakeholders}
The following stakeholders are described along with their concerns. These concerns eventually determine the key drivers of the service.

\subsection{Represented stakeholders}


\textbf{Passengers}  are the direct users of Easy Ticketing. The typical target customer is the traveler who uses the train systems regularly. Passengers are mainly concerned with the reliability and the availability of the system as well as the security often their data and the regular payment of the ticket. If the service is not available then they risk not having a ticket and eventually getting a fine.  Another concern of the passengers could be the incompatibility of their phone with the beacon, which could risk the validity of their ticket.

\textbf{Customers} are the owners of the Easy Ticketing. The typical target are the large train companies, operating in the Netherlands. The companies are mainly concerned with the reliability of the system, since many passengers could end up not paying, given that there is a breakdown of the system. This could result in monetary loss for the company. Another main concern of our customer is the security. If the system is hacked the  company loses money and this is certainly something that the company wants to avoid.

\textbf{Developers and maintenance team} is responsible for building the software part of our system, test and debug it afterwards. They are mainly concerned with the maintainability, testability and portability of the system.

\textbf{Architects} will be the team that will design the system. Their main concern is satisfying the needs of all the stakeholders and finding the best solution. They determine the feasibility of desired properties and functions, and guarantee a satisfactory end-product.


\subsection{Non represented stakeholders}


\textbf{Hardware companies}  are the companies that provide beacons for our system. In our case they are non represented stakeholders, since they do not have much say  


\textbf{Hardware maintenance team} maintains the physical side of the system, in our case beacons and takes care of handling malfunctions.

\textbf{Ticket inspectors} are the employees of the Train companies, that check the passengers, whether they have an available ticket or not. The are mostly concerned with the security and the reliability of the system.

\textbf{Third party payment systems} are companies that provide a paying system for the passengers.  


\section{Key Drivers}


In order to determine the most important key drivers for our system, the stakeholders are required to give points to the key drivers according to what they consider as the most important driver for the system. Since the weight of the decision differs among our stakeholders, they are given different amount of points. The train companies (customer) have 150, the architects have 120, the developer team has 100, and the passengers 80. Considering the distribution of the points, the top 3 will be chosen as key drivers of our system. \cref{tbl:key_drivers} below shows how the different stakeholders used there points.

\begin{table}[H]
  \centering
  \begin{tabularx}{\textwidth}{l|lllllll}
    & \rotatebox[origin=l]{75}{Security} & \rotatebox[origin=l]{75}{Reliability} & \rotatebox[origin=l]{75}{Compatibility}& \rotatebox[origin=l]{75}{Performance} & \rotatebox[origin=l]{75}{Maintainability} & \rotatebox[origin=l]{75}{Scalability} \\ \hline
   
    Train Companies(Customer)    & 50  & 50 & 20 & 30          & 0           & 0      \\
    Architects                   & 50  & 30 & 20 & 20          & 0           & 0      \\
    Developer team               &  0  & 10 & 10 & 20          & 50          & 10     \\
    Passengers                   & 40  & 10 & 30  & 0          & 0           & 0      \\
    Total                & \textbf{140} & \textbf{100}& \textbf{80}  & 70  & 50   & 10    \\
  \end{tabularx}
  \caption{Key driver selection}
  \label{tbl:key_drivers}
\end{table}


\begin{description}[align=left]
  \item[Security] is concerned with the management of possible risks that may affect our system. Since \ac{ET} is working with delicate data which involves payments and bank account records The stakeholders came to the agreement that security is the most important key driver of the system. It is mostly affected by the ability of the system to survive attacks, threats, the algorithms used, etc. The customers(train companies) and passengers are mostly concerned with security. 
  
  \item[Reliability] is measured as the probability that a system will not fail and that it will perform its intended function for a specified time interval. This can be guaranteed by  redundancy, implementing an extra database, increase in resources, implementation of more than a single point of failure for different parts of the system, etc. Passengers are mostly concerned with the reliability of the system as they don't want to risk not having a ticket, while train companies do not want to get any economical loss. 
  
  \item[Compatibility] is the ability of the software to work with other systems. In the case of \ac{ET}, the system should be able to adapt to three different types of the operating system of the smartphones: iOS, Android, Windows. The stakeholders that are mostly concerned with this, are the passangers and the train companies.  

  \item[Maintainability]  is the \cite{web:software} ability of the system to undergo changes with a degree of ease. These changes could impact components, services, features, and interfaces when adding or changing the functionality, fixing errors, and meeting new business requirements.

  \item[Performance] is concerned with how long it takes the system to respond to an event. The performance of our system is affected by:

  \begin{itemize}
    \item Latency (time between the arrival of the stimulus and the systems response to it)
    \item Deadlines in processing
    \item Throughput of the system (the number of transactions the system can process in a second)
    \item Jitter of the response (the variation in latency)
    \item Miss rate (the number of events not processed because the system was busy to respond)
    \item Data loss (data that was lost because the system was busy)
  \end{itemize}

  Within our system context, the performance which brings most interest to the stakeholders is the performance of \ac{ET} during the event itself. This is related to the detection of risky individuals, matching them to their profiles in the database and fight detection.
  The stakeholders that are mostly concerned about performance are the architects and the customers.

\item[Scalability] is ability \cite{web:df} of a system to either handle increases in load without impact on the performance of the system, or the ability to be readily enlarged. 

\end{description}




\section{Functional requirements}

{
  \renewcommand{\arraystretch}{1.5}
  \begin{table}[H]
    \centering
    \begin{tabularx}{\linewidth}{l|l|X}
     	FR-1 & Must & There should be at least 2 different ways of payment. \\ \hline
     	FR-2 & Must & The beacons shall connect to a phone's bluetooth and application.\\ \hline
     	FR-3 & Must & The system MUST/WILL/HAVE TO charge a person based on account/travel location. \\
     	\hline
     	FR-4 & Must & The system should verify which cart sends which data (or train actually). \\
     	\hline
     	FR-5 & Must & Phones should be verified based on accounts.\\ \hline        
     	FR-6 & Must & The system should provide a back-up to log out of the trip in case battery dies, via touch-screen panels in stations.\\ \hline
     	FR-7 & Must & The beacons will start connecting to mobile devices once the train is 300m away from the station.\\ \hline
    	FR-8 & Must & Phone application should have beacon verification.\\	\hline
	    FR-9 & Must & The train will collect data from the carts/beacons and send them to the server  after leaving a station. \\ 	\hline
	    FR-10 & Must & The system should charge on trip exit.\\ 	\hline
    	FR-11 & Must & Once a connection between a beacon and a phone is established, the system must send a notification to the user’s phone.\\ 	\hline
	    FR-12 & Must & Application provides a user interface enabling the user to check his account, update amounts.
        \\	\hline
	    FR-13 & Must & The beacons register both logged in and not logged in users and match the account with the phone even if the user logs in later on during the journey. \\ \hline
      FR-13 & Must & The payment should be secure. \\

    \end{tabularx}
    \caption{Functional requirements}
  \end{table}
}

\section{Non-functional requirements}

\subsection{Security}
{
  \renewcommand{\arraystretch}{1.5}
  \begin{table}[H]
    \centering
    \begin{tabularx}{\textwidth}{l|l|X}
      NF-1.1 &Must & One account is allowed to login at one smartphone at one time.\\ \hline
      NF-1.2 &Must & Users will need to authenticate in order to login. \\ \hline
      NF-1.3 &Must & The data information provided by the costumers will be safe and secure. \\ \hline
      NF-1.4 &Must & Every request from a subserver component to the main repository must be verified so that the repository knows at all times who sent the request. \\  
    \end{tabularx}
  \end{table}
}

\subsection{Reliability}
{
  \renewcommand{\arraystretch}{1.5}
  \begin{table}[H]
    \centering
    \begin{tabularx}{\textwidth}{l|l|X}
      NF-2.1 &Must & The databases must be available 99.9999\% of the time during a train journey.\\ \hline
      NF-2.2 &Must & The system must be fault tolerant.  \\ \hline
      NF-2.3 &Must & The beacons inside the carts will be operational at all times when the train is working. \\ \hline
      NF-2.4 &Must & The Server/database will be 99.999\% available , with backups . \\  
    \end{tabularx}
  \end{table}
}

\subsection{Compatibility}
{
  \renewcommand{\arraystretch}{1.5}
  \begin{table}[H]
    \centering
    \begin{tabularx}{\textwidth}{l|l|X}
      NF-3.1 &Must & The system should be compatible with the 3 main os (ios, android, windows). \\ \hline
      NF-3.2 &Must & Compatible frequencies. \\ 
     
    \end{tabularx}
  \end{table}
}


%\section{Constraints}

%A constraint is \cite{web:bus-doc} considered as an element, factor, or subsystem that works as a bottleneck. It restricts an entity, project, or system (such as a manufacturing or decision making process) from achieving its potential (or higher level of output) with reference to its goal. We are going to analyze the constraints of our systems in three different categories: organisational, business and technological.

%\subsection{Organisational Constraints}
%\subsection{Business Constraints}
%\subsection{Technological Constraints}

%\section{Risk Assessment}
%%!TEX root = ./main.tex
An assessment of the risks considered in the architectural process are presented in the table below 
{
  \renewcommand{\arraystretch}{0.8}
  \begin{sidewaystable}
    \centering
    \caption{Risk assessment}
    \label{tbl:risk_assessment}
    \begin{tabularx}{\textwidth}{XXXXXXX}
      \textbf{Risk} & \textbf{Impact, odds} & \textbf{Responsibility} & \textbf{Threshold} & \textbf{Consequences} & \textbf{Prevention} & \textbf{Reaction} \\ \hline
      \multicolumn{7}{l}{\textit{Business}} \\ \hline
      Poor response to the final product hence few people use it after spending money and time developing it & High, medium & Management, Architects & Poor response to the product due to unawareness or due to the final product, which customers find complicated to use & Project could be scrapped hence a waste of the funds put in it's development. & Advertise the product sufficiently to improve awareness and knowledge of the product, and ensure a sound architecture that is favourable for the users & Implement promotions for the users of the product to raise curiosity and interest \\ \hline
      Customer dissatisfaction  & Medium, medium & Architects, software developers & The final product is underwhelming and customers find it more difficult or the same as using the formal methods of payment & Little or no customers use the product hence, making it redundant & Develop a sound product to provide an easy payment option for it's customers &Attempt to improve the product at as low a cost as is possible before deciding to scrap it \\ \hline
      \multicolumn{7}{l}{\textit{Functionality}} \\ \hline
      User's phone battery dies before they have arrived at their final destination & High, High & Architect, Management & The continuous use of one's smart phone during a long journey in addition to the power used by the bluetooth to keep the connection to the beacons can lead to one's battery dying during the journey. & Mis-calculation of final travel fee & Make charging ports available in trains with this system or make the system available only in trains that have charging ports. & Message notification sent to user when battery is running out. \\ \hline
    \end{tabularx}
  \end{sidewaystable}
}
{
\renewcommand{\arraystretch}{1.2}
  \begin{sidewaystable}
    \begin{tabularx}{\textwidth}{XXXXXXX}
      \textbf{Risk} & \textbf{Impact, odds} & \textbf{Responsibility} & \textbf{Threshold} & \textbf{Consequences} & \textbf{Prevention} & \textbf{Reaction} \\ \hline
      \multicolumn{7}{l}{\textit{Functionality} (cont.)} \\ \hline
      The beacon connects to a phone of a user who has already used the conventional form of payment. & Medium, High & Architect & A beacon is trying to establish a connection with a user's smart phone, which has the account activated and yet the user has already paid for the journey. & System charges the user again leading to the user being inconvenienced & Notification sent to the user of an incoming connection to the service and an option of whether to accept the connection. & If user has notification feature switched off, they can access the application and cancel the journey stating that they have already paid, which can be confirmed by providing a ticket ID \\ \hline
      Hacker uses own beacon the try and access a user's phone. & High, medium & Architect, Developer & A hacker with a beacon similar to the ones in the train tries to connect to users' phones in order to receive the payments or other information. & User's private information on their device can be accessed and stolen & The train beacons and the phone application exchange keys generated by the system and the user's phone cannot be accessed without this key. & The application notifies the user that it cannot identify the beacon that is trying to connect to it. \\ \hline
      \multicolumn{7}{l}{\textit{Technology}} \\ \hline
      Low bluetooth range for a beacon & High, Medium & Architect & A user moves to a different train cart, which is out of range of the beacon it is connected to, which can lead a disconnection . & More than one beacon is used per cart in the train . & Another beacon closer to the user connects to their device. \\ \hline
    \end{tabularx}
  \end{sidewaystable}
}