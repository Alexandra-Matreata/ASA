%!TEX root = main.tex
This chapter will provide information regarding the baseline architectural decisions taken for implementing the Easy Ticketing system. The first section will outline the main tactics used related to the key drivers and section \ref{chp:design_decisions} will describe the main architectural decisions.
 
\section{Tactics}
This section outlines the tactics used for implementing the three most important quality attributes for the system.

\subsection{Compatibility Tactics}
In this section, the tactics to ensure the system's compatibility are presented with a reason behind the use of this particular tactic.
\begin{itemize}
\item \textbf{Bluetooth beacons} 
The use of bluetooth beacons for the system to interact with mobile devices ensures compatibility in a way that bluetooth is available to almost all mobile devices and on top of that, it also takes advantage of backward compatibility so any version of bluetooth on any device will be compatible with the bluetooth version of the beacons.

\item \textbf{Mobile application}
The implementation of mobile applications for the various selected operating systems that is iOS, Android and Windows ensures the compatibility of the system with almost all mobile devices given that these are the dominant operating systems in the smartphone OS market share according to the International Data Corporation (IDC) \cite{web:IDC}.

\item \textbf{Mobile application backwards compatibility}
Backwards compatibility ensures that a newer version of the mobile application works with the older version. This would ensure that if a user has not updated to the latest mobile application software version, they can still use the application. This is to be implemented for only one version below the newer version since more than that would make it hard to maintain the application. 
\end{itemize}
\subsection{Security Tactics}

Below we are briefly describing the security tactics applied for our system.
\begin{itemize} 

\item \textbf{Authenticate users}
Authentication is ensuring that a user or remote computer is actually who it purports to be.In our system (ET), every user of the mobile app needs to be authorized by inserting their personal info data. 

\item \textbf{Authorize users}
Authorization is ensuring that an authenticated user has the rights to access and modify either data or services. This is usually managed by providing some access control patterns within a system. Easy Ticketing has implemented a system, where the user will receive a notification by the time they establish a bluetooth connection with the beacon.

\item \textbf{Maintain data confidentiality} 
Data should be protected from unauthorized access. Confidentiality is usually achieved by applying some form of encryption to data and to communication links. Encryption provides extra protection to persistently maintained data beyond that available from authorization. 

\item \textbf{Maintain integrity} Data should be delivered as intended. It can have redundant information encoded in it, such as checksums or hash results, which can be encrypted either along with or independently from the original data.

\item \textbf{Limit exposure} Attacks typically depend on exploiting a single weakness to attack all data and services on a host. The architect can design the allocation of services to hosts so that limited services are available on each host.

\item \textbf{Limit access} Firewalls restrict access based on message source or destination port. Messages from unknown sources may be a form of an attack. It is not always possible to limit access to known sources. In our case Easy Ticketing  expects requests from mobile app users and from third party payment systems.  One configuration used in our case is the so-called demilitarized zone (DMZ). 
\end{itemize} 



\subsection{Reliability Tactics}
The following tactics implemented for increasing the reliability of the system have been extracted from \cite{tactics-reliability}.

\begin{itemize}

\item \textbf{Ping-Echo}
The Ping-Echo mechanism is used for checking the availability of a specific component or part of the application. A component sends a ping request to the component we wish to monitor at periodic intervals. If the monitored component fails to reply, it can be deducted that the respective component is unresponsive or that a failure occurred. In this case, maintenance operations may be started for that particular component.

This tactic helps improve reliability by notifying operators of failures in the system in a timely manner, which will decrease the time needed for maintenance operations and improve the system's time between failures and overall availability. 

For the case of the Easy Ticketing system, this tactic has been implemented for monitoring the activity of the main server component.

\item \textbf{Raising Exceptions}
This tactic needs to be implemented at component level and for as many components as possible. Each component should be able to detect specific faults or errors regarding the system's behaviour in certain conditions and raise an exception in order to notify its presence.

By implementing this tactic, the system's fault tolerance is increased proportionally to the number of exceptions being raised(and handled respectively).

In the case of the Easy Ticketing system, this tactic has been implemented for all main components(beacons, local servers, main server, mobile application). The specific exceptions raised by each component, as well as the way in which these are handled will be further explained in chapter \ref{chp:soft_arch}.

\item \textbf{Active or passive redundancy}
Resource redundancy is one of the most common tactics used for increasing the reliability of a system. Although it introduces performance drawbacks and it increases the effort needed for maintenance, resource redundancy helps deal with failure at a component level and greatly increases the availability and fault tolerance of the system.

Resource redundancy has been introduced as follows:
\begin{enumerate}
	\item Passive redundancy for the main server component(in connection to the master-slave replication pattern described in chapter \ref{chp:patterns})
	
	\item Active redundancy for the local components inside the train carts(beacons and local servers) as part of the replicated systems pattern described in chapter \ref{chp:patterns}
\end{enumerate}

\end{itemize}

\section{Design decisions}
\label{chp:design_decisions}
%!TEX root = main.tex

In this section, the major design decisions made about our system are discussed, the reasoning behind their selection is presented with their possible alternatives if available.
\begin{table}[H]
	\centering
	\begin{tabularx}{\linewidth}{l|X}
		\textbf{Name}      & Bluetooth beacons \\ \hline
		\textbf{Decision ID}  & DD1 \\ \hline
		\textbf{Alternative(s)}     & Train wifi  \\ \hline
		\textbf{Description}   & Bluetooth beacons are hardware transmitters that communicate with nearby portable devices. Each cart shall have at least 2 beacons making the connections to the users' devices strong and reliable. \\ \hline
		\textbf{Rationale}    & The use of bluetooth beacons over wi-fi was selected because in addition to it ensuring no compatibility issues, the beacons are also able to identify the position of the connected devices so, this helps with the pricing of the tickets that is between first class and second class if the system can know where a passenger is seated. This also helps the system know when a person leaves the train. \\ \hline
	\end{tabularx}
	\caption{Design Decisions - DD1}
	\label{tbl:uc1}
\end{table}

\begin{table}[H]
	\centering
	\begin{tabularx}{\linewidth}{l|X}
		\textbf{Name}      & Mobile Applications\\ \hline
		\textbf{Decision ID}  & DD2 \\ \hline
		\textbf{Alternative(s)}     & Web application \\ \hline
		\textbf{Description}   & Mobile applications shall be developed for the main operating systems that is Android, IOS and Windows. More emphasis shall be put on the development of applications for Android and IOS given that they have the largest market shares meaning more users are likely to have devices with this operating system.\\ \hline
		\textbf{Rationale}    & The use of a mobile application makes this system easy for the users to access and use. The mobile application also ensures the security of the system and the user since both the beacon and the device have to exchange security information before they can start exchanging other information. \\ \hline
	\end{tabularx}
	\caption{Design Decisions - DD2}
	\label{tbl:uc1}
\end{table}

\begin{table}[H]
	\centering
	\begin{tabularx}{\linewidth}{l|X}
		\textbf{Name}      & Local server\\ \hline
		\textbf{Decision ID}  & DD3 \\ \hline
		\textbf{Alternative(s)}     & NA \\ \hline
		\textbf{Description}   & This will be where information about the users' travel information will be temporarily stored before it is sent off to the central server for processing. This component will have at least 50gb storage available to it.\\ \hline
		\textbf{Rationale}    & This component ensures there is no data lost during a journey given that the long journeys could reach areas without access to the internet hence losing the connection to the central server and if the local server is not available, it will lead to data loss. \\ \hline
	\end{tabularx}
	\caption{Design Decisions - DD3}
	\label{tbl:uc1}
\end{table}

\begin{table}[H]
	\centering
	\begin{tabularx}{\linewidth}{l|X}
		\textbf{Name}      & Central server \\ \hline
		\textbf{Decision ID}  & DD4 \\ \hline
		\textbf{Alternative(s)}     & NA  \\ \hline
		\textbf{Description}   & This is the final data storage point and where the calculation of the journey prices are calculated. The central server will consist of the database and computational units. It shall also be connected to an external payment system where the train journeys will be securely paid for. \\ \hline
		\textbf{Rationale}    & The central server is the most vital component of the system and without it, the system wouldn't be able to work.  \\ \hline
	\end{tabularx}
	\caption{Design Decisions - DD4}
	\label{tbl:uc1}
\end{table}

\begin{table}[H]
	\centering
	\begin{tabularx}{\linewidth}{l|X}
		\textbf{Name}      & Mobile app notifications \\ \hline
		\textbf{Decision ID}  & DD5 \\ \hline
		\textbf{Alternative(s)}     & NA  \\ \hline
		\textbf{Description}   & The mobile application shall notify the users when a beacon connects to their device and to it.\\ \hline
		\textbf{Rationale}    & This feature ensures that if the application is on and the user has already paid for a ticket, they don't end up paying again through the system. \\ \hline
	\end{tabularx}
	\caption{Design Decisions - DD5}
	\label{tbl:uc1}
\end{table}

\begin{table}[H]
	\centering
	\begin{tabularx}{\linewidth}{l|X}
		\textbf{Name}      & Cryptographic key identification \\ \hline
		\textbf{Decision ID}  & DD6 \\ \hline
		\textbf{Alternative(s)}     &  NA \\ \hline
		\textbf{Description}   & The Local server and central server shall communicate with each other after the local server identifies itself using a cryptographic key.\\ \hline
		\textbf{Rationale}    & This feature ensures the security of the central server as it will have vital data and should not be able to be accessed by any other unknown external systems. \\ \hline
	\end{tabularx}
	\caption{Design Decisions - DD6}
	\label{tbl:uc1}
\end{table}