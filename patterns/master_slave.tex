%!TEX root = ../main.tex

\section{Master Slave}

	The master slave replication pattern introduces passive resource redundancy by allowing a back-up(slave) component to remain on stand-by until the master component fails, at which point it can take over the work load until the main component is repaired.

	\subsubsection{Problem}

		In case of a failure at the level of the main central server, the system can no longer perform until the error is repaired. A back-up solution is needed in order to keep the system available even in case of such a failure.

	\subsubsection{Solution} 

		Introduce a back-up server component able to perform the same functionalities as the main one and using the same or a similar database which is updated in real time(no differences between the two databases should be noticeable at any point).
		
	\subsubsection{Rationale}
	
	The back-up server component can monitor the main server using a Ping/Echo signal determining its state while remaining in a passive mode(only the database is updated in real time, while all other requests and computations are performed by the main component). Once the main component fails to respond, the slave component takes over the work load until the main component is repaired.
	


\paragraph{Implications}
\begin{itemize}
	\item increase reliability
	\item slight decrease in performance
	\item increase maintainability effort
	
\end{itemize}
