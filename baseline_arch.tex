%!TEX root = main.tex
This chapter will provide information regarding the baseline architectural decisions taken for implementing the Easy Ticketing system. The first section will outline the main tactics used related to the key drivers and section \ref{chp:design_decisions} will describe the main architectural decisions.
 
\section{Tactics}
This section outlines the tactics used for implementing the three most important quality attributes for the system.

\subsection{Compatibility Tactics}
In this section, the tactics to ensure the system's compatibility are presented with a reason behind the use of this particular tactic.
\begin{itemize}
\item \textbf{Bluetooth beacons} 
The use of bluetooth beacons for the system to interact with mobile devices ensures compatibility in a way that bluetooth is available to almost all mobile devices and on top of that, it also takes advantage of backward compatibility so any version of bluetooth on any device will be compatible with the bluetooth version of the beacons.

\item \textbf{Mobile application}
The implementation of mobile applications for the various selected operating systems that is iOS, Android and Windows ensures the compatibility of the system with almost all mobile devices given that these are the dominant operating systems in the smartphone OS market share according to the International Data Corporation (IDC) \cite{web:IDC}.
\end{itemize}
\subsection{Security Tactics}

Below we are briefly describing the security tactics applied for our system.
\begin{itemize} 

\item \textbf{Authenticate users}
Authentication is ensuring that a user or remote computer is actually who it purports to be.In our system (ET), every user of the mobile app needs to be authorized by inserting their personal info data. 

\item \textbf{Authorize users}
Authorization is ensuring that an authenticated user has the rights to access and modify either data or services. This is usually managed by providing some access control patterns within a system. Easy Ticketing has implemented a system, where the user will receive a notification by the time they establish a bluetooth connection with the beacon.

\item \textbf{Maintain data confidentiality} 
Data should be protected from unauthorized access. Confidentiality is usually achieved by applying some form of encryption to data and to communication links. Encryption provides extra protection to persistently maintained data beyond that available from authorization. 

\item \textbf{Maintain integrity} Data should be delivered as intended. It can have redundant information encoded in it, such as checksums or hash results, which can be encrypted either along with or independently from the original data.

\item \textbf{Limit exposure} Attacks typically depend on exploiting a single weakness to attack all data and services on a host. The architect can design the allocation of services to hosts so that limited services are available on each host.

\item \textbf{Limit access} Firewalls restrict access based on message source or destination port. Messages from unknown sources may be a form of an attack. It is not always possible to limit access to known sources. In our case Easy Ticketing  expects requests from mobile app users and from third party payment systems.  One configuration used in our case is the so-called demilitarized zone (DMZ). 
\end{itemize} 



\subsection{Reliability Tactics}
The following tactics implemented for increasing the reliability of the system have been extracted from \cite{tactics-reliability}.

\begin{itemize}

\item \textbf{Ping-Echo}
The Ping-Echo mechanism is used for checking the availability of a specific component or part of the application. A component sends a ping request to the component we wish to monitor at periodic intervals. If the monitored component fails to reply, it can be deducted that the respective component is unresponsive or that a failure occurred. In this case, maintenance operations may be started for that particular component.

This tactic helps improve reliability by notifying operators of failures in the system in a timely manner, which will decrease the time needed for maintenance operations and improve the system's time between failures and overall availability. 

For the case of the Easy Ticketing system, this tactic has been implemented for monitoring the activity of the main server component.

\item \textbf{Raising Exceptions}
This tactic needs to be implemented at component level and for as many components as possible. Each component should be able to detect specific faults or errors regarding the system's behaviour in certain conditions and raise an exception in order to notify its presence.

By implementing this tactic, the system's fault tolerance is increased proportionally to the number of exceptions being raised(and handled respectively).

In the case of the Easy Ticketing system, this tactic has been implemented for all main components(beacons, local servers, main server, mobile application). The specific exceptions raised by each component, as well as the way in which these are handled will be further explained in chapter \ref{chp:soft_arch}.

\item \textbf{Active or passive redundancy}
Resource redundancy is one of the most common tactics used for increasing the reliability of a system. Although it introduces performance drawbacks and it increases the effort needed for maintenance, resource redundancy helps deal with failure at a component level and greatly increases the availability and fault tolerance of the system.

Resource redundancy has been introduced as follows:
\begin{enumerate}
	\item Passive redundancy for the main server component(in connection to the master-slave replication pattern described in chapter \ref{chp:patterns})
	
	\item Active redundancy for the local components inside the train carts(beacons and local servers) as part of the replicated systems pattern described in chapter \ref{chp:patterns}
\end{enumerate}

\end{itemize}

\section{Design decisions}
\label{chp:design_decisions}
%!TEX root = main.tex
