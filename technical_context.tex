The system will need to perform in a specifically designed technical context consisting of three main environments: a central server collecting all information and performing necessary computations, the setup inside the trains themselves (comprised of beacons and small servers which will collect information per train and send it to the central one every time the train leaves the station) and a login system in each station allowing users to manually check-out of their trip and perform payment.

A set of measures will need to be taken into consideration for the system to be able to operate inside this context by using these different components and environments. These measures will be related to the key attributes required of the system as follows:
\begin{itemize}

\item \textbf{security}: since the system will have access to sensitive information (such as location, financial transactions), every connection between components of the system will need to be secured and trusted. The main connections which will need to be verified each time they are created will be: beacon to mobile application, beacon to server inside the train, small local server to central server. 

\item \textbf{reliability}: for the system to be considered reliable, the main components should be provided with a back-up in case of failure, such as the central server, the beacons in each cart, etc. Also, users should be presented with alternative scenarios in case the main desired activity flow gets interrupted (e.g. possibility to manually check-out of a trip using login system in train station in case of beacon to phone communication being severed)

\item \textbf{compatibility}: the system will need to be compatible with at least the main and most commonly used mobile OS and frequencies 

\end{itemize}